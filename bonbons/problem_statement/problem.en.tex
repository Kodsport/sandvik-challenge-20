\problemname{Bonbons}
Ylva loves bonbons, probably more than anything else on this planet.
She loves them so much that she made a large plate of $R \cdot C$ bonbons for her \emph{fikaraster}\footnote{``Fikarast'' is a Swedish word, meaning to take a break from work while enjoying coffee and pastries together with your colleagues.}.

Ylva has a large wooden tray which can fit $R$ rows of $C$ bonbons per row, that she will put the bonbons on.
Her bonbons have three different fillings: Nutella Buttercream, Red Wine Chocolate Ganache, and Strawberry Whipped Cream.
Since Ylva is a master chocolatier, she knows that presentation is $90\%$ of the execution.
In particular, it looks very bad if two bonbons of the same color are adjacent to each other within a row or a column on the tray.
We call an arrangement of bonbons where this is never the case a \emph{good arrangement}.

Given the number of bonbons of each flavour, and the size of Ylva's tray, can you help her find a good arrangement of the bonbons, or determine that no such arrangement exists?

\section*{Input}
The first line of input contains the two space-separated integers $2 \le R, C \le 1000$.
The next line contains three non-negative space-separated integers $a, b, c$ -- the number of bonbons of the three flavours which Ylva has baked.
It is guaranteed that $a + b + c = R \cdot C$.
Both $R$ and $C$ will be even.

\section*{Output}
If no good arrangement can be found, output \texttt{impossible}.
Otherwise, output $R$ lines, each containing $C$ characters, representing a good arrangement.
Each row should contain only characters \texttt{A, B, C}, depending on which flavour should be placed on a certain position.
The number of \texttt{A} bonbons placed must be equal to $A$, and so on.
